\documentclass[10pt,a4paper,twocolumn,landscape]{article}

\usepackage[english,russian]{babel}
\usepackage{amsmath,amssymb,mathspec}
\usepackage{soul}
\usepackage{pgf,tikz,pgfplots}
\usepackage{wrapfig,graphicx}
\usepackage{fancyhdr}
\usepackage{multirow,multicol,enumitem}
\usepackage[top=2cm, bottom=2cm, left=2cm, right=1.8cm]{geometry}

% Set fonts
%\setmainfont{Times New Roman}
\setmainfont[Scale=1]{Liberation Serif}
%\setmathrm[Scale=0.8]{Cascadia Code PL} % for \sin, \cos, ...
\usepackage[euler-digits,euler-hat-accent]{eulervm}
\usepackage[mathscr]{euscript}
\setallsansfonts{DejaVu Sans}
\setallmonofonts[Scale=0.95]{Cascadia Code PL}

\setlist[enumerate]{nosep}

\everymath{\displaystyle}
\newcommand{\Z}{\mathcal{Z}}
\newcommand{\Q}{\mathcal{Q}}
\newcommand{\N}{\mathcal{N}}
\newcommand{\No}{\mathcal{N}_0}
\newcommand{\R}{\mathcal{R}}

\pgfplotsset{compat=newest}



\begin{document}

\setlength{\parindent}{0cm}

\pagestyle{fancy}
\fancypagestyle{plain}{}
\setlength{\headheight}{15pt}
\fancyhead[L]{\it TODO}

\center{\textbf{Часть 1}}

\framebox[11.5cm]{
	\begin{minipage}[c]{11.2cm}
		\textit{\textbf{
			Ответами к заданиям 1--23 являются число или последовательность
			цифр. Ответ запишите в поле ответа в тексте работы, а затем
			перенесите в БЛАНК ОТВЕТОВ № 1 справа от номера
			соответствующего задания, начиная с первой клеточки. Кждый символ
			пишите в отдельной клеточке в соответствии с приведёнными в бланке
			образцами. Единицы измерения физических величин писать не нужно.
		}}
	\end{minipage}
}

\hspace{-1.1cm}
\begin{tabular}{c l}
	\framebox[0.85cm]{\textbf{1}} &
	\begin{minipage}[t]{11.5cm}
		Материальная точка равномерно движется по окружности радиусом $R$
		со скоростью $v$. Во сколько раз нужно увеличить скорость её движения,
		чтобы при увеличении радиуса окружности в 4 раза центростремительное
		ускорение точки осталось прежним?

		Ответ: в \underline{\hspace{4.5cm}} раз(а).
	\end{minipage}
\end{tabular}

\end{document}
