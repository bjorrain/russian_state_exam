\documentclass[10pt,a4paper,twocolumn,landscape]{article}

\usepackage[english,russian]{babel}
\usepackage{amsmath,amssymb,mathspec}
\usepackage{soul}
\usepackage{pgf,tikz,pgfplots}
\usepackage{wrapfig,graphicx}
\usepackage{fancyhdr,lastpage,array}
\usepackage{multirow,multicol,enumitem}
\usepackage[top=2.5cm, bottom=2cm, left=2.2cm, right=1.5cm]{geometry}

% Set fonts
%\setmainfont{Times New Roman}
\setmainfont[Scale=1]{Liberation Serif}
%\setmathrm[Scale=0.8]{Cascadia Code PL} % for \sin, \cos, ...
\usepackage[euler-digits,euler-hat-accent]{eulervm}
\usepackage[mathscr]{euscript}
\setallsansfonts{DejaVu Sans}
\setallmonofonts[Scale=0.95]{Cascadia Code PL}

\setlist[enumerate]{nosep}

\everymath{\displaystyle}
\newcommand{\Z}{\mathcal{Z}}
\newcommand{\Q}{\mathcal{Q}}
\newcommand{\N}{\mathcal{N}}
\newcommand{\No}{\mathcal{N}_0}
\newcommand{\R}{\mathcal{R}}

\pgfplotsset{compat=newest}
\setlength{\parindent}{0cm}

\setlength{\columnsep}{2.5cm}
\setlength{\columnwidth}{11.3cm}

\renewcommand{\headrulewidth}{0cm}
\renewcommand{\footrulewidth}{0cm}
\renewcommand{\headruleskip}{0.6cm}
\renewcommand{\footruleskip}{0cm}

\newcommand{\PreserveBackslash}[1]{\let\temp=\\#1\let\\=\temp}
\newcolumntype{C}[1]{>{\PreserveBackslash\centering}p{#1}}
\newcolumntype{R}[1]{>{\PreserveBackslash\raggedleft}p{#1}}
\newcolumntype{L}[1]{>{\PreserveBackslash\raggedright}p{#1}}


\title{Открытый вариант}
\author{ФИЗИКА}


\begin{document}

\makeatletter
\pagestyle{fancy}
\fancypagestyle{plain}{}
\setlength{\headheight}{0cm}
\fancyhead[L]{\textsf{\scriptsize 
	\begin{tabular}{L{4.2cm} R{4.2cm} R{1.9cm}}
		\hspace{-0.1cm}\@title & \@author & \thepage/\pageref{LastPage}
	\end{tabular}
}}
\fancyhead[R]{\textsf{\scriptsize \@title\hspace{0.4cm}}}
\fancyfoot[L]{
	\hspace{0.5cm}
	\begin{minipage}[l]{11.5cm}
		\center{
			\textsf{
				\tiny
				(C) 2023 Федеральная служба по надзору в сфере образования и науки
				\\[-0.2cm]
				Копирование \textbf{не допускается}
			}
		}
	\end{minipage}
}
\fancyfoot[C]{}
\makeatother


\center{\textbf{Часть 1}}

\hspace{-0.25cm}\framebox[11.5cm]{
	\begin{minipage}[l]{11.2cm}\noindent
		\textit{\textbf{
			Ответами к заданиям 1--23 являются число или последовательность
			цифр. Ответ запишите в поле ответа в тексте работы, а затем
			перенесите в БЛАНК ОТВЕТОВ № 1 справа от номера
			соответствующего задания, начиная с первой клеточки. Кждый символ
			пишите в отдельной клеточке в соответствии с приведёнными в бланке
			образцами. Единицы измерения физических величин писать не нужно.
		}}
		\vspace{-0.1cm}
	\end{minipage}
}

\vspace{1cm}

\hspace{-1.5cm}
\begin{tabular}{c l}
	\framebox[0.85cm]{\textbf{1}}\hspace{-0.15cm} &
	\begin{minipage}[t]{11.25cm}
		Материальная точка равномерно движется по окружности радиусом $R$
		со скоростью $v$. Во сколько раз нужно увеличить скорость её движения,
		чтобы при увеличении радиуса окружности в 4 раза центростремительное
		ускорение точки осталось прежним?

		Материальная точка равномерно движется по окружности радиусом $R$
		со скоростью $v$. Во сколько раз нужно увеличить скорость её движения,
		чтобы при увеличении радиуса окружности в 4 раза центростремительное
		ускорение точки осталось прежним?

		Материальная точка равномерно движется по окружности радиусом $R$
		со скоростью $v$. Во сколько раз нужно увеличить скорость её движения,
		чтобы при увеличении радиуса окружности в 4 раза центростремительное
		ускорение точки осталось прежним?

		Материальная точка равномерно движется по окружности радиусом $R$
		со скоростью $v$. Во сколько раз нужно увеличить скорость её движения,
		чтобы при увеличении радиуса окружности в 4 раза центростремительное
		ускорение точки осталось прежним?

		Материальная точка равномерно движется по окружности радиусом $R$
		со скоростью $v$. Во сколько раз нужно увеличить скорость её движения,
		чтобы при увеличении радиуса окружности в 4 раза центростремительное
		ускорение точки осталось прежним?

		Материальная точка равномерно движется по окружности радиусом $R$
		со скоростью $v$. Во сколько раз нужно увеличить скорость её движения,
		чтобы при увеличении радиуса окружности в 4 раза центростремительное
		ускорение точки осталось прежним?
		
		Материальная точка равномерно движется по окружности радиусом $R$
		со скоростью $v$. Во сколько раз нужно увеличить скорость её движения,
		чтобы при увеличении радиуса окружности в 4 раза центростремительное
		ускорение точки осталось прежним?

		\vspace{1em}Ответ: в \underline{\hspace{4.5cm}} раз(а).
	\end{minipage}
\end{tabular}


\center{\textbf{Часть 1}}

\hspace{-0.25cm}\framebox[11.5cm]{
	\begin{minipage}[l]{11.2cm}\noindent
		\textit{\textbf{
			Ответами к заданиям 1--23 являются число или последовательность
			цифр. Ответ запишите в поле ответа в тексте работы, а затем
			перенесите в БЛАНК ОТВЕТОВ № 1 справа от номера
			соответствующего задания, начиная с первой клеточки. Кждый символ
			пишите в отдельной клеточке в соответствии с приведёнными в бланке
			образцами. Единицы измерения физических величин писать не нужно.
		}}
		\vspace{-0.1cm}
	\end{minipage}
}

\vspace{1cm}

\hspace{-1.5cm}
\begin{tabular}{c l}
	\framebox[0.85cm]{\textbf{1}}\hspace{-0.15cm} &
	\begin{minipage}[t]{11.25cm}
		Материальная точка равномерно движется по окружности радиусом $R$
		со скоростью $v$. Во сколько раз нужно увеличить скорость её движения,
		чтобы при увеличении радиуса окружности в 4 раза центростремительное
		ускорение точки осталось прежним?

		Материальная точка равномерно движется по окружности радиусом $R$
		со скоростью $v$. Во сколько раз нужно увеличить скорость её движения,
		чтобы при увеличении радиуса окружности в 4 раза центростремительное
		ускорение точки осталось прежним?

		Материальная точка равномерно движется по окружности радиусом $R$
		со скоростью $v$. Во сколько раз нужно увеличить скорость её движения,
		чтобы при увеличении радиуса окружности в 4 раза центростремительное
		ускорение точки осталось прежним?

		Материальная точка равномерно движется по окружности радиусом $R$
		со скоростью $v$. Во сколько раз нужно увеличить скорость её движения,
		чтобы при увеличении радиуса окружности в 4 раза центростремительное
		ускорение точки осталось прежним?

		Материальная точка равномерно движется по окружности радиусом $R$
		со скоростью $v$. Во сколько раз нужно увеличить скорость её движения,
		чтобы при увеличении радиуса окружности в 4 раза центростремительное
		ускорение точки осталось прежним?

		Материальная точка равномерно движется по окружности радиусом $R$
		со скоростью $v$. Во сколько раз нужно увеличить скорость её движения,
		чтобы при увеличении радиуса окружности в 4 раза центростремительное
		ускорение точки осталось прежним?
		
		Материальная точка равномерно движется по окружности радиусом $R$
		со скоростью $v$. Во сколько раз нужно увеличить скорость её движения,
		чтобы при увеличении радиуса окружности в 4 раза центростремительное
		ускорение точки осталось прежним?

		\vspace{1em}Ответ: в \underline{\hspace{4.5cm}} раз(а).
	\end{minipage}
\end{tabular}




\end{document}
