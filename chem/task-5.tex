Среди предложенных формул веществ, расположенных в пронумерованных ячейках, выберите формулы: А) трёхосновной кислоты; Б) двухкислотного основания; В) амфотерного оксида.\\
	
		\begin{tabular}{|c|c|c|}
		\hline
		\makecell{ 1) оксид хрома (III) } & \makecell{ 2) сероводородная \\ кислота} & \makecell{ 3) N_2O_5 }\\
		\hline
		\makecell{ 4) алюмогидрид \\ лития} & \makecell{ 2)гидроксид \\ стронция} &\makecell{ 3) метафосфорная \\ кислота }\\
		\hline
		7) CuO & 8) лимонная кислота & 9) гидроксид алюминия \\ 
		\hline
		\end{tabular} \sepline 

%неизменяемая часть
\sepline
Ответ:
\begin{tabular}{|c|c|c|}
\hline
А) & Б) & В) \\
\hline
__ & __ & __ \\
\hline
\end{tabular}
